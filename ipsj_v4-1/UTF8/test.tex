\documentclass[submit]{ipsj}

\usepackage{graphicx}
\usepackage{latexsym}

\setcounter{巻数}{1}
\setcounter{号数}{1}
\setcounter{page}{1}

\受付{2025}{12}{19}
\採録{2025}{12}{19}

\begin{document}

\title{日本語テスト:あいうえお漢字カタカナ}

\etitle{Japanese Test: Hiragana Kanji Katakana}

\affiliate{TEST}{テスト大学\\
Test University}

\author{山田 太郎}{Taro Yamada}{TEST}[test@example.com]

\begin{abstract}
これは日本語のテストです。
ひらがな、カタカナ、漢字が正しく表示されるかを確認します。
あいうえお、アイウエオ、亜伊宇江於。
\end{abstract}

\begin{jkeyword}
日本語,テスト,確認
\end{jkeyword}

\begin{eabstract}
This is a test for Japanese language display.
We check if Hiragana, Katakana, and Kanji are displayed correctly.
\end{eabstract}

\begin{ekeyword}
Japanese, test, verification
\end{ekeyword}

\maketitle

\section{日本語表示テスト}

以下の文字が正しく表示されるか確認してください:

\begin{itemize}
\item ひらがな:あいうえお かきくけこ さしすせそ
\item カタカナ:アイウエオ カキクケコ サシスセソ
\item 漢字:日本語 情報処理 学会 論文
\item 記号:〇△□ ※ ・ 「」『』
\end{itemize}

\section{本文テスト}

本研究では、日本語の表示に関する問題を解決することを目的とする。
情報処理学会の論文テンプレートを使用して、
UTF-8エンコーディングで記述された日本語文書が
正しくPDFに変換されることを確認する。

\end{document}
