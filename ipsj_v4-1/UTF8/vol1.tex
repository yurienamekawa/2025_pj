\documentclass[submit]{ipsj}

\usepackage{graphicx}
\usepackage{latexsym}

\setcounter{巻数}{1}
\setcounter{号数}{1}
\setcounter{page}{1}

\受付{2025}{12}{19}
\採録{2025}{12}{19}

\begin{document}

\title{論文タイトルをここに記入}

\etitle{English Title Here}

\affiliate{UNIV}{所属大学・機関名\\
University Name}

\author{著者 太郎}{Taro Chosha}{UNIV}[author@example.com]

\begin{abstract}
ここに和文の概要を記述します。
本研究では〇〇について検討し、〇〇という結果を得た。
\end{abstract}

\begin{jkeyword}
キーワード1,キーワード2,キーワード3
\end{jkeyword}

\begin{eabstract}
This is the English abstract.
Write your abstract here.
\end{eabstract}

\begin{ekeyword}
keyword1, keyword2, keyword3
\end{ekeyword}

\maketitle

%1
\section{はじめに}

ここに本文を記述します。

本研究の背景と目的について述べます。


%2
\section{関連研究}

関連する研究について述べます。


%3
\section{提案手法}

提案する手法について詳しく説明します。


%4
\section{実験}

実験の設定と結果について述べます。

\subsection{実験設定}

実験の設定について説明します。

\subsection{実験結果}

実験結果を示します。


%5
\section{考察}

実験結果に基づいた考察を行います。


%6
\section{おわりに}

本研究のまとめと今後の課題について述べます。


\begin{acknowledgment}
本研究の一部は〇〇の支援を受けて行われました。
\end{acknowledgment}

\begin{thebibliography}{9}
\bibitem{sample1}
著者名:論文タイトル,学会誌名,Vol.XX, No.X, pp.XX-XX (20XX).

\bibitem{sample2}
Author, A. and Author, B.: Title of the Paper, Proc. Conference Name, pp.XX-XX (20XX).
\end{thebibliography}

\begin{biography}
\profile{m}{著者 太郎}{20XX年〇〇大学卒業。20XX年同大学院修了。現在、〇〇に従事。}
\end{biography}

\end{document}
